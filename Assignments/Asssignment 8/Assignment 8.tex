\documentclass[12pt]{article}

\title{Group Homework 8}
\author{G2 - Robert Krency, Austin Pringle, Anthony Stepich}
\date{\today}

\usepackage{tikz}
\usepackage{amsmath}
\usepackage{graphicx}
\usepackage{tabularx}
\usepackage{multicol}
\usepackage{enumitem}

\usepackage[normalem]{ulem}

% Geometry 
\usepackage{geometry}
\geometry{letterpaper, left=1in, top=1in, right=1in, bottom=1in}

% Macros
\newcommand{\definition}[1]{\underline{\textbf{#1}}}

\newcommand{\encircle}[1]{%
  \tikz[baseline=(X.base)] 
    \node (X) [draw, shape=circle, inner sep=0] {\strut #1};}

% Begin Document
\begin{document}

\maketitle

\begin{enumerate}

    %1
    \item f(wxyz) = $\sum_m(8,9,10,13,15) + \sum_\delta(0,4,6,12,14)$ = WX + WZ' + WY' \\
    \begin{tabular}[t]{c c c}

        \\

        \begin{tabular}[t]{l l}

             0 & 0000 \\ \hline
             4 & 0100 \\
             8 & 1000 \\ \hline
             6 & 0110 \\ 
             9 & 1001 \\
            10 & 1010 \\
            12 & 1100 \\ \hline
            13 & 1101 \\
            14 & 1110 \\ \hline
            15 & 1111 \\

        \end{tabular}

        \hspace{8mm}

        & 

        \begin{tabular}[t]{l l}

              0,4  & 0-00 \\
              0,8  & -000 \\ \hline
              4,6  & 01-0 \\
              4,12 & -100 \\
              8,9  & 100- \\
              8,10 & 10-0 \\
              8,12 & 1-00 \\ \hline
              6,14 & -110 \\
              9,13 & 1-01 \\
             10,14 & 1-10 \\
             12,13 & 110- \\
             12,14 & 11-0 \\ \hline
             13,15 & 11-1 \\
             14,15 & 111- \\

        \end{tabular}

        \hspace{8mm}

        &

        \begin{tabular}[t]{l l}

             0,4,8,12 & - - 0 0  \\ \hline
             4,6,12,14 & - 1 - 0 \\
             8,9,12,13 & 1 - 0 - \\
             8,10,12,14 & 1 - - 0 \\ \hline
             12,13,14,15 & 1 1 - - \\

        \end{tabular}

    \end{tabular}

    \begin{tabular}{l | c | c c c c c}
            & WXYZ & 8 & 9 & 10 & 13 & 15 \\ \hline
        Y'Z' & - - 0 0 & \sout{X} & \\
        XZ'  & - 1 - 0 & \\
        WY'  & 1 - 0 - & \sout{X} & \encircle{X} & & \sout{X} & \\
        WZ'  & 1 - - 0 & \sout{X} & & \encircle{X} \\
        WX   & 1 1 - - & & & & \sout{X} & \encircle{X} \\
    \end{tabular}
    
    \pagebreak

    %2
    \item f(wxyz) = $\sum_m(5,9,14,15) + \sum_\delta(0,1,7,8,11-13)$ = WX + Y'Z \\
    \begin{tabular}[t]{c c c}

        \\

        \begin{tabular}[t]{l l}

             0 & 0000 \\ \hline

             1 & 0001 \\
             8 & 1000 \\ \hline

             5 & 0101 \\ 
             9 & 1001 \\
            12 & 1100 \\ \hline

            7  & 0111 \\
            11 & 1011 \\
            13 & 1101 \\
            14 & 1110 \\ \hline

            15 & 1111 \\

        \end{tabular}

        \hspace{8mm}

        & 

        \begin{tabular}[t]{l l}

              0,1  & 000- \\
              0,8  & -000 \\ \hline

              1,5  & 0-01 \\
              1,9  & -001 \\
              8,9  & 100- \\
              8,12 & 1-00 \\ \hline

              5,7  & 01-1 \\
              5,13 & -101 \\
              9,11 & 10-1 \\
              9,13 & 1-01 \\
             10,14 & 1-10 \\
             12,13 & 110- \\
             12,14 & 11-0 \\ \hline

             7,15  & -111 \\
             11,15 & 1-11 \\
             13,15 & 11-1 \\
             14,15 & 111- \\

        \end{tabular}

        \hspace{8mm}

        &

        \begin{tabular}[t]{l l}

             0,1,8,9     & - 0 0 - \\ \hline
             
             1,5,9,13    & - - 0 1 \\
             8,9,12,13   & 1 - 0 - \\ \hline

             5,7,13,15   & - 1 - 1 \\ 
             9,11,13,15  & 1 - - 1 \\
             12,13,14,15 & 1 1 - - \\


        \end{tabular}

        \\

    \end{tabular}

    \begin{tabular}{l | c | c c c c}
              & W X Y Z & 5 & 9 & 14 & 15 \\ \hline
        WX    & 1 1 - - & & & \encircle{X} & \sout{X} \\
        WZ    & 1 - - 1 & & \sout{X} & & \sout{X} \\ 
        XZ    & - 1 - 1 & \sout{X} & & & \sout{X} \\
        WY'   & 1 - 0 - & & \sout{X} \\
        Y'Z   & - - 0 1 & \encircle{X} & \sout{X} \\
        X'Y'  & - 0 0 - & & \sout{X} \\
    \end{tabular}
   


    \pagebreak

    %3
    \item f(wxyz) = $\sum_m(0,3,7,11,13) + \sum_\delta(2,4,6,12)$ = W'Y + W'Z' + WXY' + X'YZ\\
    \begin{tabular}[t]{c c c}

        \\

        \begin{tabular}[t]{l l}

             0 & 0000 \\ \hline

             2 & 0010 \\
             4 & 0100 \\ \hline

             3 & 0011 \\ 
             6 & 0110 \\
            12 & 1100 \\ \hline

            7  & 0111 \\
            11 & 1011 \\
            13 & 1101 \\

        \end{tabular}

        \hspace{8mm}

        & 

        \begin{tabular}[t]{l l}

              0,2  & 00-0 \\
              0,4  & 0-00 \\ \hline

              2,3  & 001- \\
              2,6  & 0-10 \\
              4,6  & 01-0 \\
              4,12 & -100 \\ \hline

              3,7  & 0-11 \\
              3,11 & -011 \\
              6,7  & 011- \\
             12,13 & 110- \\

        \end{tabular}

        \hspace{8mm}

        &

        \begin{tabular}[t]{l l}

             0,2,4,6     & 0 - - 0 \\ \hline
             
             2,3,6,7     & 0 - 1 - \\

        \end{tabular}

        \\

    \end{tabular}

    \begin{tabular}{l | c | c c c c c}
              & W X Y Z & 0 & 3 & 7 & 11 & 13 \\ \hline
        W'Z'  & 0 - - 0 & \encircle{X} \\
        W'Y   & 0 - 0 - & & \sout{X} & \encircle{X} \\
        XY'Z' & - 1 0 0 & \\
        X'YZ  & - 0 1 1 & & \sout{X} & & \encircle{X} \\
        WXY'  & 1 1 0 - & & & & & \encircle{X} \\
    \end{tabular} 



    \pagebreak

    %4
    \item f(wxyz) = $\sum_m(2,8,9,10-12) + \sum_\delta(3,6,13-15)$ = W + YZ' = W + X'Y\\
    \begin{tabular}[t]{c c c}

        \\

        \begin{tabular}[t]{l l}

             2 & 0010 \\
             8 & 1000 \\ \hline

             3 & 0011 \\ 
             6 & 0110 \\
             9 & 1001 \\
            10 & 1010 \\
            12 & 1100 \\ \hline

            11 & 1011 \\
            13 & 1101 \\
            14 & 1110 \\ \hline

            15 & 1111 \\

        \end{tabular}

        \hspace{8mm}

        & 

        \begin{tabular}[t]{l l}

              2,3  & 001- \\
              2,6  & 0-10 \\
              2,10 & -010 \\
              8,9  & 100- \\
              8,12 & 1-00 \\ \hline

              3,11 & -011 \\
              6,14 & -110 \\
              9,13 & 1-01 \\
             10,11 & 101- \\
             10,14 & 1-10 \\
             12,13 & 110- \\
             12,14 & 11-0 \\ \hline

             11,15 & 1-11 \\
             13,15 & 11-1 \\
             14,15 & 111- \\

        \end{tabular}

        \hspace{8mm}

        &

        \begin{tabular}[t]{l l}

             2,3,10,11 & - 0 1 - \\
             2,6,10,14 & - - 1 0 \\
             8,9,12,13 & 1 - 0 - \\
             8,9,10,11 & 1 0 - - \\
             9,11,13,15 & 1 - - 1 \\
             12,13,14,15 & 1 1 - - \\ \hline

             8,9,10,11,12,13,14,15 & 1 - - - \\

        \end{tabular}

        \\

    \end{tabular}

    \begin{tabular}{l | c | c c c c c c}
              & W X Y Z & 2 & 8 & 9 & 10 & 11 & 12 \\ \hline
        X'Y   & 0 - - 0 & \encircle{X} & & \sout{X} & \sout{X} \\
        YZ'   & 0 - 0 - & \encircle{X} & & \sout{X} \\
        WY'   & - 1 0 0 & & \sout{X} & & & & \sout{X} \\
        WZ    & - 0 1 1 & & & \sout{X} & & \sout{X} \\
        W     & 1 - - - & & \encircle{X} & \sout{X} & \sout{X} & \sout{X} & \sout{X} \\
    \end{tabular} 
   

\end{enumerate}

\end{document}